\documentclass[a4paper,zihao=-4,UTF8]{ctexart}
\pagestyle{plain}
\date{}
\usepackage[top=1.0in, bottom=1.0in, left=1.25in, right=1.25in]{geometry} % set margins for A4 paper
\usepackage[bodytextleadingratio=1.67]{zhlineskip} % use line skip optimized for Chinese text
\usepackage{zhnumber} % for converting number to Chinese numbering
\usepackage{abstract} % for abstract

\usepackage{indentfirst} % for indenting the first paragraph of each section
\setlength{\parindent}{2em} % set indent for each paragraph

\usepackage{graphicx} % for including images

\usepackage{hyperref} % for setting hyper-ref styles
\hypersetup{
    colorlinks=true, % set true if you want colored links
    linktoc=all,     % set to all if you want both sections and subsections linked
    linkcolor=black,  % we want all links to be black
    urlcolor=black,
    citecolor=black
}

\usepackage{booktabs} % for better table rules
\usepackage{multirow} % for multirow cells in tables
\usepackage{enumitem} % for better control of itemize and enumerate
\usepackage{amsmath,amssymb} % for mathematical symbols
\usepackage{algorithm} % for algorithm environment
\usepackage{algpseudocode} % for algorithm pseudocodes
\usepackage{etoolbox} % for setups in environments

% numbering restarts at each section
\numberwithin{figure}{section}  
\numberwithin{table}{section}
\numberwithin{equation}{section}
\numberwithin{algorithm}{section}

\usepackage{xcolor,colortbl} % for colorful tables and better color mixing
\usepackage[capitalize]{cleveref}  % for setting cross-referencing names
\crefname{figure}{图}{图}
\crefname{table}{表}{表}
\crefname{equation}{公式}{公式}
\crefname{algorithm}{算法}{算法}


% convenient commands for changing font
\newcommand{\hei}[1]{\heiti\sffamily #1}
\newcommand{\song}[1]{\songti\rmfamily #1}


% set up style for table of contents
\usepackage{titlesec} % for setting styles of section titles
\usepackage{titletoc} % for setting styles of TOC
\renewcommand{\contentsname}{\zihao{2} \heiti 目录}
\addtocontents{toc}{\vspace{\baselineskip}}
\titlecontents{section}[0em]
  {\zihao{-4}\bfseries\heiti}
  {\contentspush{第\zhnumber{\thecontentslabel}章\quad}}
  {\hspace{0em}}
  {\titlerule*[2pt]{$.$}\contentspage}
\titlecontents{subsection}[2em]
  {\zihao{-4}}{\contentslabel{2em}}
  {\hspace{-2em}}
  {\titlerule*[0.2pc]{$.$}\contentspage}
\titlecontents{subsubsection}[4em]
  {\zihao{-4}}
  {\contentslabel{3em}}
  {\hspace{-3em}}
  {\titlerule*[0.2pc]{$.$}\contentspage}

% update references to Chinese style, font size and type can also be set here
\usepackage{caption} % for tuning caption styles
\titleformat{\section}{\centering\hei\zihao{-2}}{第\chinese{section}章}{1em}{}[]
\titleformat{\subsection}{\song\zihao{3}\bfseries}{\arabic{section}.\arabic{subsection}}{1em}{}[]
\titleformat{\subsubsection}{\song\zihao{-3}\bfseries}{\arabic{section}.\arabic{subsection}.\arabic{subsubsection}}{1em}{}[]
\renewcommand{\abstractnamefont}{\hei\zihao{-2}}
\renewcommand{\abstracttextfont}{}

\renewcommand{\figurename}{图}
\renewcommand{\tablename}{表}
\renewcommand{\thefigure}{\arabic{section}-\arabic{figure}}
\renewcommand{\theequation}{\arabic{section}.\arabic{equation}}
\renewcommand{\thetable}{\arabic{section}-\arabic{table}}
\renewcommand{\thealgorithm}{\arabic{section}-\arabic{algorithm}} 
\makeatletter
\renewcommand{\ALG@name}{算法} % change title prefix in algorithm environment
\makeatother
\captionsetup{labelsep=space}


% title text
\newcommand{\mytitle}{番茄炒蛋}

% header
\usepackage{fancyhdr}
\pagestyle{fancy}

\makeatletter
\newcommand{\if@sectionstar}[2]{ % check whether we are in a \section* or \section
	\ifnum\value{@instarsection}=0
		{#1}
	\else
		{#2}
	\fi
}
\renewcommand{\sectionmark}[1]{\markboth{\if@sectionstar{第\chinese{section}章~}{}#1}{}} % only add prefix "第x章" in header for sections without star (i.e. \section*).
\makeatother

\fancyhead[L]{\song{\zihao{-5}\mytitle}}
\fancyhead[R]{\song{\zihao{-5}\leftmark}}
\renewcommand\headrulewidth{.5pt}
\renewcommand\footrulewidth{0pt}

% for hyper-reference of "section*" to work properly
\usepackage{xparse}
\let\oldsection\section
\makeatletter
\newcounter{@secnumdepth}
\newcounter{@instarsection}
\RenewDocumentCommand{\section}{s o m}{%
  \IfBooleanTF{#1}
    {\setcounter{@secnumdepth}{\value{secnumdepth}}% Store secnumdepth
     \setcounter{secnumdepth}{0}% Print only up to \chapter numbers
     \setcounter{@instarsection}{1} % mark that we are in a \section*. See \if@sectionstar.
     \oldsection{#3}% \section*
     \setcounter{secnumdepth}{\value{@secnumdepth}}}% Restore secnumdepth
    {\setcounter{@instarsection}{0} % mark that we are in a \section. See \if@sectionstar.
     \IfValueTF{#2}% \section
       {\oldsection[#2]{#3}}% \section[.]{..}
       {\oldsection{#3}} % \section{..}
    }
}
\makeatother

% use GB/T 7714—2015 BibTeX Style
\usepackage{gbt7714}
\bibliographystyle{gbt7714-numerical}
\citestyle{numbers} % use number "[1]" instead of super "^{[1]}"

% set the separate spacing for lists like enum, item
\setlist{
  noitemsep,
  topsep=0pt,
  parsep=0pt,
  partopsep=0pt,
  leftmargin=*
}

% set the separate spacing for bibs
\setlength{\bibsep}{0pt}

% set the font size in tables
\DeclareCaptionFont{tablefont}{\zihao{5}}
\captionsetup[table]{font=tablefont}
\AtBeginEnvironment{tabular}{\zihao{5}}
